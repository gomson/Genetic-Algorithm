\documentclass[12pt]{article}
\usepackage[hmargin=2cm,vmargin=2cm]{geometry}
\usepackage{ntheorem,picinpar}
\usepackage{ctexcap}
\usepackage{fontspec,multirow}
% \setmainfont{Adobe Caslon Pro}
\usepackage{amsmath,bm}
\usepackage{amsfonts}
\usepackage{xcolor,tabularx,booktabs}
\usepackage{graphicx}
\usepackage{fancyhdr}
\usepackage{subfigure}
\usepackage{paralist,float}
\usepackage[section]{placeins}
\pagestyle{fancy}
\usepackage[bf,md,up,raggedright]{titlesec}
\usepackage{hyperref}
\usepackage{booktabs}
\hypersetup{colorlinks,breaklinks, urlcolor=blue, linkcolor=magenta,citecolor=blue}
\makeatletter
\newcommand{\rmnum}[1]{\romannumeral #1}
\newcommand{\Rmnum}[1]{\expandafter\@slowromancap\romannumeral #1@}                                   
\makeatother
\usepackage{titlesec}
\usepackage{tabularx}
\titleformat{\section}{\scshape\Large\raggedright}{\Rmnum{\thesection}.}{1ex}{}
\hypersetup{colorlinks,breaklinks, urlcolor=blue, linkcolor=black,citecolor=blue}
\title{遗传算法的~\textsc{Python}~实现\\
Genetic Algorithm in \textsc{Python}}
\author{\kaishu{}杨元辉\quad 学号:5090209074\thanks{杨元辉,学号5090209074,上海交通大学,机械与动力工程学院,机械电子工程}\\
\textsc{Yuanhui Yang}\quad Student ID: 5090209074}
\date{\kaishu{}上海交通大学\\\textsc{Shanghai Jiao Tong University}}
\begin{document}
\lfoot{\kaishu 上海交通大学~SJTU}
\cfoot{}
\rfoot{\thepage}
\maketitle
\thispagestyle{fancy}
\section{遗传算法简介}
遗传算法是一门边缘交叉学科,是生物、数学、计算机理论等多个领域的完美结合,是生物进化淘汰在二进制计算机领域的出色实现。
\subsection{生物学基础}
生物世界充满了淘汰、竞争、突变等进化活动。以自然选择学说为核心的现代生物进化理论,其基本观点是:
\begin{itemize}
\item 种群是生物进化的基本单位,
\item 生物进化的实质是种群基因频率的改变,
\item 具体实现是:
\begin{itemize}
\item 基因突变:产生生物进化的原材料I;
\item 基因重组:产生生物进化的原材料II;
\item 自然选择:使种群的基因频率定向改变并决定生物进化的方向;
\item 生殖隔离:新物种形成的必要条件。
\end{itemize}
\end{itemize}\par
相应地,在计算机领域,上述过程的计算机实现,就是遗传算法的具体实现。
\section{计算机实现}
\subsection{编码}
遗传算法的编码规则有二进制编码与浮点数编码两种,在下面的算例中使用的是二进制编码。这样操作的优势是:二进制编码及符合计算机处理信息的原理,也方便对染色体进行遗传、编译和突变等操作,因为二进制数组相对应十进制数组更容易实现简并。
\begin{equation}
\delta =  \dfrac{U - L}{2^k - 1}
\end{equation}
\subsection{解码}
将二进制数组转变为十进制数组的过程就称为解码过程。
\begin{equation}
x = L + \left(\sum\limits_{i = 1}^{k}{b_k 2^{i-1}}\right)\cdot \dfrac{U-L}{2^k - 1} 
\end{equation}
\subsection{交配}
\begin{enumerate}
\item 产生一个0,1之间的随机数,若该数小于0.1(交配概率),则进行以下操作;
\item 用随机数(整数)产生一个交配位点;
\item 将两个二进制数组在以上交配位点之后的所有值互换。
\end{enumerate}
\subsection{突变}
\begin{enumerate}
\item 产生一个0,1之间的随机数,若该数小于0.4(突变概率),则进行以下操作;
\item 用随机数(整数)产生一个突变位点;
\item 将二进制数组上的突变位点0,1互换,具体实现是:
\begin{equation}
value = abs\left(value-1\right)
\end{equation}
\end{enumerate}
\subsection{自然选择}
模拟自然选择法则的过程,我们使用轮盘选择法,即以高概率选中结果更加出色的解,作为下一轮计算的样本。
\begin{enumerate}
\item 解码;
\item 计算对应的目标函数值;
\item 将以上函数值归一化到$\left(0,1\right)$区间,同时加和为1,即频率;
\item 计算以上归一化结果的积累频率;
\item 产生一个$\left(0,1\right)$之间的随机数,将该结果对应的二进制数组提出,这就是我们选出来的相对最优解;
\item 将该结果代入下一轮循环,如此下去,结果相对佳的解便在解集中扩散开去,结果不佳的解会被慢慢地被淘汰。
\end{enumerate}\par
以上过程是随机的,充满概率的,按照解出色的程度进行概率抽取,也给其他并不是最出色的解机会,这样做是为了防止陷入局部最优解。
\section{算例}
\begin{equation}\label{eqn:dec}
\begin{array}{l}
\min \left\{ {\exp \times \left( x \right)\left( {4{x^2} + 2{y^2} + 4xy + 2y + 1} \right)} \right\}\\
\mathrm{subject~to:}\left\{ \begin{array}{l}
1.5 + xy - x - y \le 0\\
 - xy \le 10
\end{array} \right.
\end{array}
\end{equation}
\subsection{具体实现}
\subsubsection{预处理}
遗传算法之中有使用概率,概率非负。因此,需要做变换:
\begin{equation}\left\{
\begin{array}{l}
z  \gets z + \mathrm{Intial~Value}\\
z>0
\end{array}\right.
\end{equation}\par
遗传算法是求取全局最大值的,而式~\eqref{eqn:dec}~是求取最小值的。因此,需要做变换:
\begin{equation}
z\gets \dfrac{1}{z}
\end{equation}
\subsubsection{边界处理}
非常明显,经过预处理之后,以上问题的对应的函数值一定大于0,因此,我们只需要将超过边界部分的解对应的函数值强行设定成0。这样的话,在“自然选择”这步,0就会被屏蔽掉,对应的解便会被淘汰掉。
\begin{equation}
\mathrm{When:}\left\{
\begin{array}{l}
1.5 + xy - x - y > 0\\
 - xy > 10
 \end{array}\right.
 \quad \mathrm{then:}
 z\gets 0
\end{equation}
\subsubsection{实验结果}
经过500次训练之后,我们得到该NP问题可接受的全局最优解,即最后一个训练结果\texttt{VarMax[-1]}:
\begin{equation}
\left[x,y,f\left(x,y\right)\right]= \left[-6.3045 , 1.5527 ,0.2353 \right]
\end{equation}\par
另一方面,我们对上述解作边界检测,看一看是否满足边界条件:
\begin{table}[H]
\centering
\caption{边界检测}
\begin{tabular}{lcr}
\toprule
计算项目&$1.5+xy - x - y \le 0$&$-xy\le 10$
\\\midrule
计算结果&-3.5377 &  9.7895\\
满足与否&是&是
\\\bottomrule
\end{tabular}
\end{table}
% \par
% 使用~\textsc{Matlab}~作图,我们能够看到程序输出的平均值与最大值的变化趋势是越来越接近,这说明训练样本在“进化”。
% \begin{figure}[H]
% \centering
% \includegraphics[width=\textwidth]{fig01.eps}
% \caption{平均值与最大值}
% \end{figure}
% \section{总结}
% 经过遗传算法的求解,我们成功地解决了该非线性回归问题,具体的结果满足边界调节,求解的结果也可以接受。\par
% 足以见到,遗传算法的结果是一种可以信赖的算法,它基于对之前求解过程的学习。相比穷举算法,遗传算法在解决NP问题时有着独特的优势。
% \section{鸣谢}
% 最后,感谢陆老师在程序尾声为我提供的帮助,老师面向对象以及解决问题的视角对产生了积极的影响。在老师的ftp中我看到了数据库等资料,这是我进一步学习的重要参考。\par
% 之后我即将从机械行业转到计算机科学,我选择北京微软亚洲研究院作为我职业生涯的起点,希望有朝一日能够成为一位出色的计算机科学家。由衷地感谢老师。\par
% 全文使用~\LaTeX~编译完成,一方面说明我对cs的浓厚兴趣,一方面也说明我的职业或学术选择。
\end{document}